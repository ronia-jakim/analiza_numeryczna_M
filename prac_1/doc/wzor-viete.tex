\section{Wzór Viete'a}

% https://en.wikipedia.org/wiki/Vi%C3%A8te%27s_formula

Viete wyprowadził swoją formułę na $\pi$ obserwując stosunek pola $2^n$-kata foremnego do pola $2^{n+1}$-kata foremnego. Poprzez zwiększanie $n$ w nieskończoność, jesteśmy w stanie dostać stosunek $2^2$-kąta foremnego, czyli kwadratu, do pola koła w które został on wpisany. Pole koła o promieniu $1$ wynosi
$$P_k=\pi$$
a pole kwadratu w nie wpisanego:
$$P_2={2\cdot 2\over 2}=2.$$
W takim razie stosunek pola kwadratu do pola koła wynosi:
\begin{align*}
    {P_2\over P_k}&={2\over\pi}=\lim_{k\to\infty}{2\over P_{2^k}}=\lim_{k\to\infty}{2\over P_{2^3}}{P_{2^3}\over 2^k}=\lim_{k\to\infty}\prod\limits_{i=1}^k{a_{n}\over a_{n+1}}=\lim_{k\to\infty}{a_1\over a_k}=\lim_{k\to\infty}{2\over a_k},
\end{align*}
gdzie  $a_k$ to pole $2^k$-kąta foremnego, z $a_1=2$.




$$\pi=\lim\limits_{k\to\infty}2^k\sqrt{2-a_k}$$
$$a_1=0$$
$$a_k=\sqrt{2+a_{k+1}}$$