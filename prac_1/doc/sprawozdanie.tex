\documentclass[11pt, wide, leqno]{mwart}

\usepackage{../../template}
%\usepackage{showframe}

\tit{P.1.6.}

\begin{document}
\maketitle
\tableofcontents

\section{Wstęp}\label{sec:ws}

W 1897 roku Amerykański fizyk Edward J. Goodwin oznajmił światu, że udało mu się poprawnie skonstruować kwadrat o polu równym polu koła. W tym samym roku jego wynik został przedłożony Zebraniu Stanu Indiana i w życie wszedł Artykuł o Liczbie Pi, na mocy którego $\pi=3.2$. Ponad 2000 lata wcześniej, w III wieku p.n.e., Archimedes napisał
$$3.1408\approx{223\over71}<\pi<{22\over7}\approx 3.1428,$$
co daje ograniczenia bliższe tym dzisiaj uważanym za najdokładniejsze niż artykuł z XIX wieku. W 2009 roku za pomocą algorytmu braci Chudowskych, wyprowadzonego ze wzorów S. Ramanujana, zostało osiągnięte przybliżenie $\pi$ przez 10 trylionów cyfr znaczących. Dzisiaj istnieją jeszcze szybsze algorytmu na przybliżanie $\pi$ niż ten z 2009 roku i badanie metod przybliżania wartości matematycznych jest nadal żywą dziedziną nauki.

W tej pracy przyjrzymy się 5 metodom przybliżania liczby $\pi$. W Rozdziale 2. zaprezentujemy naiwne algorytmy, z których jedna okazała się nie odstawać od bardziej zaawansowanych sposobów zaprezentowanych w dalszych rozdziałach. W Rozdziałach 3 i 4 przyjrzymy się dwóm metodom liniowo zbieżnym do $\pi$. W Rozdziale 5. zaprezentowany jest algorytm zbieżny kwadratowo do liczby $\pi$.

\subsection{Metoda}

Do obliczeń używaliśmy precyzji 16 069 i zmiennych typu \verb+BigFloat+ w języku \verb+Julia+. Za wartość dokładną $\pi$ użyliśmy wartości bibliotecznej typu \verb+Irrational+. Dzięki temu jest ona wyliczana dla każdej precyzji \cite{julia}.

Dla każdej metody eksperymentalnie szacowaliśmy rząd zbieżności i przedstawiliśmy wyniki na odpowiednich wykresach oraz wyliczaliśmy liczbę dokładnie wyznaczonych cyfr liczby pi. Przy pierwszych 4 metodach wykonaliśmy 10 000 iteracji. W metodzie Chudnowskych zaś wystarczyło tylko 450 iteracji, by osiągnąć granice precyzji. Zaś w metodzie Gaussa-Legendre'a wystarczyło tylko 21 iteracji. 

\subsection{Wyznaczanie wykładnika zbieżności metod}

Jako wykładnik zbieżności metody uznajemy $p$ spełniające następujący wzór \cite{bog}

\[\lim_{n \to \infty} \frac{|a_n - \alpha|}{|a_{n-1} - \alpha|^p} = c  \]

Gdzie $a_n$ to wynik danej metody dla i-tej iteracji, $\alpha$ to wartość do której zbiega metoda czyli $\pi$, w tej pracy przyjmiemy że to wartość biblioteczna. Zaś $c$ to stała. Zachodzi następująca
\[ \lim_{n \to \infty} \frac{|a_n - \pi|}{|a_{n-1} - \pi|^p} = c = \lim_{n \to \infty} \frac{|a_{n-1} - \pi|}{|a_{n-2} - \pi|^p} \]
W takim razie 
\[ \frac{|a_n - \pi|}{|a_{n-1} - \pi|^p} \approx \frac{|a_{n-1} - \pi|}{|a_{n-2} - \pi|^p} \]

\[ (\frac{|a_{n-2} - \pi|}{|a_{n-1} - \pi|})^p \approx \frac{|a_{n-1} - \pi|}{|a_{n} - \pi|} \]
\[ p \approx \frac{\log(|\frac{a_{n-2} - \pi}{a_{n-1} - \pi}|)}{\log(|\frac{a_{n-1} - \pi}{a_{n} - \pi}|)}
\]

\section{Pierwsze próby}

\subsection{Szereg Taylora}
W matematyce bardzo często w celu przybliżania porządanych wartości używa się szeregów Taylora. Tak dla przykładu, korzystając z rozszerzenia funkcji $\arctan x$ w punkcie $0$ możemy oszacować wartość $\frac\pi4$:
\begin{equation}
\begin{split}
    \frac\pi4&=\arctan1=\sum\limits_{k=0}^\infty{\arctan^{(k)}0\over k!}(1-0)^k=\\
    &=1-\frac13+\frac15-\frac17+...=\sum\limits_{k=0}^\infty{(-1)^k\over 2k+1}.
\end{split}
\end{equation}

W obliczeniach praktycznych nie możliwe jest dodawanie kolejnych elementów sumy w nieszkończoność. Konieczne jest więc zatrzymanie się na pewnym $N$, co daje pewien błąd, $R_N$:
$$\frac\pi4\approx \sum\limits_{k=0}^N{(-1)^k\over 2k+1}+R_N.$$
Oznaczmy tę sumę jako $P_N$. Ponieważ dla przybliżeń funkcji szeregiem Taylora coraz wyższego stopnia dostajemy coraz dokładniejszy wynik, to $P_{N+1}$ powinno być dokładniejsze niż $P_N$. Zauważamy też, że
$$P_{N+1}-P_N={(-1)^{N+1}\over 2N+3}$$
w takim razie możemy oszacować błąd dla szeregu Taylora $N$-tego stopnia za pomocą
$$R_N\approx \max{(-1)^{N+1}\over 2N+3},$$
co daje zbieżność liniową.

Problem tego przybliżenia $\pi$ został przeanalizowany już przez Madhawa z Sangamagramy w XIV wieku. Zaproponował on następującą korekcję wzoru dla skończonych sum:
\begin{equation}
    \frac\pi4\approx \sum\limits_{k=0}^N{(-1)^k\over 2k+1}\pm{N^2+1\over 4N^3+5N}.
\end{equation}

{\color{cyan}WYPADAŁOBY NAKLEPAĆ I PRZEDSTAWIĆ WYNIKI}


\subsection{Algorytm Monte Carlo}

Ponieważ $\pi$ jest stosunkiem pola koła jednostkowego do jego promienia, do przybliżania jego wartości można skorzystać z kwadratu i ćwiartki koła. Zauważmy, że jeżeli będziemy wybierać losowo punkty kwadratu o polu 1, to $\frac\pi4$ z nich powinno znaleźć się w ćwiartce koła o środku w jednym z wierzchołków tego kwadratu:

\begin{center}
\begin{tikzpicture}
    \coordinate (A) at (0,0);
    \coordinate (B) at (4,0);
    \coordinate (C) at (4,4);
    \coordinate (D) at (0,4);

    \filldraw[fill=ziel!40!white, draw=ziel] (B) arc[start angle=0, end angle=90, radius=4];
    \filldraw[color=ziel!40!white] (A)--(B)--(D)--cycle;
    \draw[thick] (A)--(B)--(C)--(D)--cycle;

    \node at (1.5, 2) {\Large$\frac\pi4$};
    \node at (2, -0.3) {\large$1$};
    \node at (-0.3, 2) {\large$1$};
\end{tikzpicture}
\end{center}

Korzystając z algorytmu Monte Carlo możemy wybierać losowo współrzędne $x,y\in[0,1]$ kolejnych punktów, a następnie sprawdzać ile z nich spełnia warunek
$$x^2+y^2\leq1.$$
Otrzymany stosunek będzie coraz bliższy $\frac\pi4$ wraz ze zwiększaniem ilości testowanych punktów.

{\color{cyan}NAKLEPAĆ I TYM LOGIEM PRZYBLIŻYĆ ZBIEŻNOŚĆ CZY INNE CHUJU MUJU}

\subsection{Wzór Wallisa?}
9.4 ze skryptu szwarca do analizy I
$$\sqrt\pi=\lim_{n\to\infty}{(n!)^24^n\over(2n)!\sqrt n}$$

\section{Wzór Viete'a}

% https://en.wikipedia.org/wiki/Vi%C3%A8te%27s_formula

Viete wyprowadził swoją formułę na $\pi$ obserwując stosunek pola $2^n$-kata foremnego do pola $2^{n+1}$-kata foremnego. Poprzez zwiększanie $n$ w nieskończoność, jesteśmy w stanie dostać stosunek $2^2$-kąta foremnego, czyli kwadratu, do pola koła w które został on wpisany. Można ją też wyprowadzić za pomocą tożsamości udowodnionej przez Eulera ponad 100 lat po śmierci Viete'a.

Wzór zaproponowany przez Viete'a, uznawany za prekursor analizy matematycznej w matematyce poprzez pierwsze wykorzystanie nieskończonego ilorazu, jest następujący:
\begin{equation}
    {2\over\pi}=\prod\limits_{k=1}^n{a_k\over2},
\end{equation}
gdzie $a_1=\sqrt2$ oraz
$$a_k=\sqrt{2+a_{n-1}}.$$

Wiemy, że
$${\sin x\over x}=\cos\frac x2\cos\frac x4...=\lim_{n\to\infty}\prod\limits_{k=1}^n\cos{x\over 2^n}$$
oraz
$$\cos\frac x2=\sqrt{{1+\cos x\over 2}}.$$
Jeśli wstawimy $x=\frac\pi2$ i oznaczymy $b_k=\cos{x\over 2^k}$, $b_1={\sqrt2\over2}$,  dostaniemy
\begin{align*}
    {\sin\frac\pi2\over\frac\pi2}&=\frac2\pi=\lim_{n\to\infty}\prod\limits_{k=1}^n\cos{x\over 2^2}=\lim_{n\to\infty}\prod\limits_{k=1}^nb_k=\lim_{n\to\infty}\prod\limits_{k=1}^n{a_k\over 2},
\end{align*}
gdzie 
$$a_k=2b_k=2\sqrt{{1+b_{k-1}\over 2}}=\sqrt{2+2b_{k-1}}=\sqrt{2+a_{k-1}}$$
i $a_1=\sqrt2$.

\subsection{Wyniki}

Na Wykresie~\ref{viete-error}. zaprezentowany jest wykres zbieżności algorytmu Viete'a. Eksperymentalne wyznaczenie rzędu zbieżności, tak jak i gradient prezentowanego wykresu, sugerują liniową zbieżność tej metody wyliczania Viete. Od około 8750 iteracji wartość zwracana przez metodę Viete'a pokrywa się z wartością wyliczaną bibliotecznie.

\begin{figure}[!h]\centering
    \renewcommand{\figurename}{Wykres}
    \includegraphics[width=0.7\textwidth]{../prog/viete_log_error.png}
    \caption{Wykres logarytmu dziesiętnego z błędu względnego dla przybliżenia $\pi$ za pomocą metody Viete'a.}
    \label{viete-error}
\end{figure}

Zbieżność liniowa jest dobrym wynikiem, tak samo jak w podejściu geometrycznym opisanym w Sekcji~\ref{geometric-interpretation}

\section{Algorytm Chudnowsky'ch}

Algorytm zaproponowany przez braci Chudnowskych opiera się na 17 wzorach na $\frac1\pi$ opracowanych przez Srinivasa Ramanujan\cite{review}:
$${\frac {1}{\pi }}={\frac {1}{426880{\sqrt {10005}}}}\sum _{k=0}^{\infty }{\frac {(6k)!(13591409+545140134k)}{(3k)!(k!)^{3}(-640320)^{3k}}}.$$
Z tego można uzyskać $\pi$ wprost w formie wzoru:
$$\pi=C\Big(\sum\limits_{q=0}^\infty{M_q\cdot L_q\over X_q}\Big)^{-1},$$
gdzie 
$$
\begin{cases}
    C=426880\sqrt{10005}\\
    L_{q+1}=L_q+545140134 \quad L_0=13591409  \\
    X_{q+1}=X_q\cdot(-262537412640768000)\quad X_0=1\\
    K_{q+1}=K_q+12\quad K_0=-6\\
    M_{q+1}=M_{q}\cdot \left({\frac {K_{q+1}^{3}-16K_{q+1}}{\left(q+1\right)^{3}}}\right)\quad M_{0}=1.
\end{cases}
$$
Aby wyprowadzić ten wzór, jak i inne podane przez Ramanujana, potrzebna jest znajomość między innymi teorii funkcji eliptycznych\cite{review}. Z tego też względu w tym reporcie nie podejmiemy się uzasadniania poprawności wyżej podanego wzoru. Dla zainteresowanych polecamy lekturę "Collected Papers of Srinivasa Ramanujan"\cite{ramanujan}.

\begin{figure}[!h]
    \centering
    \renewcommand{\figurename}{Wykres}
    \includegraphics[width=0.7\textwidth]{../prog/chudnowsky_log_error.png}
    \caption{Wykres ilości cyfr znaczących uzyskanych dla  przybliżenia  $\pi$ za pomocą algorytmu braci Chudnowskych.}
    \label{chudnowsky-error}
\end{figure}

\begin{figure}[!h]
    \centering
    \renewcommand{\figurename}{Wykres}
    \includegraphics[width=0.7\textwidth]{../prog/chudnowsky_error_ratio.png}
    \caption{Wykres ilorazu błędów względnych wyrazu $n+1$ i $n$ dla algorytmu Chudowskych.}
    \label{chudnowsky-convergence}
\end{figure}

\subsection{Wyniki}

Logarytm z błędu względnego algorytmu Chudnowskych dla pierwszych 400 iteracji został zaprezentowany na Wykresie~\ref{chudnowsky-error}. Już dla 359 iteracji błąd bezwzględny jest równy 0. Każe to sugerować, że to właśnie ta metoda została użyta jako implementacja funkcji \verb+pi()+ w języku \verb+Julia+. Dlatego od 359 iteracji nie ma zaznaczonej wartości $\log_{10}$ od błędu względnego. Powoduje to anomalie widoczne na wykresach.

Eksperymentalne wyznaczanie zbieżności metody Chudnowskych sugeruje zbieżność liniowa, tak jak na Wykresie~\ref{chudnowsky-convergence}.. Nietypowe załamanie w okolicach 359 iteracji jest spowodowane zerową wartością błędu bezwzględnego w tym miejscu. Z pozostałej części wykresu możemy wydedukować, że
$$\lim_{k\to\infty}{{|x_{k+1}-\pi|\over |x_k-\pi|}}\approx 6.5\cdot 10^{-15},$$
gdzie $x_k$ to oszacowanie $\pi$ uzyskane w $k$-tej iteracji.

\section{Algorytm Gaussa-Legendre'a}

Algorytm Gaussa-Legendre'a jest aktualnie jednym z najszybciej zbiegających algorytmów używanych do wyliczania liczb $\pi$. Został wyprowadzony na podstawie prac Carla Friedricha Gaussa oraz Adrien-Marie Legendre na podstawie współczesnych algorytmów do mnożenia i pierwiastkowania.Jest on, niestety, bardzo wymagający pamięciowo. Poniżej prezentujemy implementację tego algorytmu\cite{gausse2}:

\newpage

\begin{lstlisting}[language=ps]
function gauss_legrendre (max):
    a = 1
    b = 1 / sqrt(2)
    t = 1 / 4
    p = 1
    i = 0
    while i <= max:
        an = (a + b) / 2
        b = sqrt(a * b)
        t = t - p * (a - an) * (a - an)
        p = 2 * p
        a = an
    
    return (a + b) * (a + b) / (4 * t)
\end{lstlisting}

\subsection{Wyniki}

Metoda Gaussa-Legrendre'a okazała się zbiegać do implementacji bibliotecznej funkcji \verb+pi()+ z języka \verb+Julia+ wyjątkowo szybko, bo już w 11 iteracji kwadrat błędu maszynowo był równy zeru, co widać na Wykresie~\ref{gauss-error}.

\begin{figure}[!h]
    \centering
    \renewcommand{\figurename}{Wykres}
    \includegraphics[width=0.7\textwidth]{../prog/gauss_legendre_log_error.png}
    \caption{Wykres logarytmu dziesiętnego z błędu względnego dla przybliżenia $\pi$ za pomocą algorytmu Gaussa-Legendre'a.}
    \label{gauss-error}
\end{figure}

\begin{figure}[!h]
    \centering
    \renewcommand{\figurename}{Wykres}
    \includegraphics[width=0.7\textwidth]{../prog/gauss_legendre_error_ratio.png}
    \caption{Wykres ilorazu błędów względnych wyrazu $n+1$ i $n$ dla algorytmu Gaussa-Legrendre'a.}
    \label{gauss-convergence}
\end{figure}

Eksperymentalne obliczenia rzędu zbieżności tej metody jedynie potwierdzają wyższą zbieżność tego algorytmu niż w przypadku innych opisanych metod (Wykres~\ref{gauss-convergence}). Dla precyzji wynoszącej 16 069 bitów obliczenia na podstawie dzielenia błędu $(n+1)$-ego wyrazu przez kwadrat błędu $n$-tego wyrazu nie dają konkretnych wyników przez zbyt szybkie dążenie tej metody do $\pi$. W literaturze metoda ta jest określana jako zbieżna kwadratowo \cite{gausse-smth}.

Pomimo tak dobrej zbieżności, metoda ta nie jest powszechnie wykorzystywana, gdyż zużywa więcej pamięci niż metoda Chudnowskych.

\newpage

Wartość $\pi$ obliczona dla 10 iteracji naszego programu daje:

{\scriptsize
3.14159265358979323846264338327950288419716939937510582097494459230781640628620899862803482534211706798214808651\\
32823066470938446095505822317253594081284811174502841027019385211055596446229489549303819644288109756659334461284756\\
4823378678316527120190914564856692346034861045432664821339360726024914127372458700660631558817488152092096282925409\\
171536436789259036001133053054882046652138414695194151160943305727036575959195309218611738193261179310511854807446\\
2379962749567351885752724891227938183011949129833673362440656643086021394946395224737190702179860943702770539217176\\
2931767523846748184676694051320005681271452635608277857713427577896091736371787214684409012249534301465495853710507\\
9227968925892354201995611212902196086403441815981362977477130996051870721134999999837297804995105973173281609631859\\
50244594553469083026425223082533446850352619311881710100031378387528865875332083814206171776691473035982534904287554\\
68731159562863882353787593751957781857780532171226806613001927876611195909216420198938095257201065485863278865936153\\
38182796823030195203530185296899577362259941389124972177528347913151557485724245415069595082953311686172785588907509\\
83817546374649393192550604009277016711390098488240128583616035637076601047101819429555961989467678374494482553797747\\
26847104047534646208046684259069491293313677028989152104752162056966024058038150193511253382430035587640247496473263\\
91419927260426992279678235478163600934172164121992458631503028618297455570674983850549458858692699569092721079750930\\
29553211653449872027559602364806654991198818347977535663698074265425278625518184175746728909777727938000816470600161\\
45249192173217214772350141441973568548161361157352552133475741849468438523323907394143334547762416862518983569485562\\
09921922218427255025425688767179049460165346680498862723279178608578438382796797668145410095388378636095068006422512\\
52051173929848960841284886269456042419652850222106611863067442786220391949450471237137869609563643719172874677646575\\
73962413890865832645995813390478027590099465764078951269468398352595709825822620522489407726719478268482601476990902\\
64013639443745530506820349625245174939965143142980919065925093722169646151570985838741059788595977297549893016175392\\
84681382686838689427741559918559252459539594310499725246808459872736446958486538367362226260991246080512438843904512\\
44136549762780797715691435997700129616089441694868555848406353422072225828488648158456028506016842739452267467678895\\
25213852254995466672782398645659611635488623057745649803559363456817432411251507606947945109659609402522887971089314\\
566913686722874894056010150330861792868092087476091782493858900971490967598526136554978189312978482168299894872...
}

\newpage

\section{Podsumowanie}

Po pierwsze, metoda zaproponowana w treści zadania jest nieoptymalna. Zbiega do $\pi$ w sposób nad liniowy, podczas gdy pozostałe metody, z pominięciem algorytmu Monte Carlo, zbiegają co najmniej liniowo. Metoda Gaussa-Legendre'a zbiega najszybciej, jednak to metoda Chudnowskych stanowi najlepszy kompromis między szybkością osiągania kolejnych liczb znaczących a zużywaną pamięcią. Co jednak najważniejsze, ludolfina została określona ze złożonością obliczeniową $O(1)$: 

\begin{center}
    \emph{Następnie sporządził odlew "morza" o średnicy dziesięciu łokci, okrągłego, o wysokości pięciu łokci i o obwodzie trzydziestu łokci} \cite{biblia}. 
\end{center}
Ponieważ $\pi={L\over 2r}={30\over 10}=3$. W takim razie wszystkie nasze obliczenia okazują się niepoprawne.

\koniec

\begin{thebibliography}{69}
    \bibitem{review}
        N. D. Baruah, B. C., Berndt, and H. H. Chan.,
        \textit{Ramanujan's Series for $\frac1\pi$: A Survey.}, 
        The American Mathematical Monthly 116, no. 7 (2009): 567-87,
        \url{http://www.jstor.org/stable/40391165}.
    \bibitem{julia}
        S. Byrne, L. Benet and D. Sanders,
        \textit{Some fun with $\pi$ in Julia},
        accessed 20.11.2022,
        \url{https://julialang.org/blog/2017/03/piday/}
    \bibitem{biblia}
        P. Bóg,
        \textit{Biblia Tysiąclecia}, 1 Krl 7, 23,
        Wydawnictwo Palottinum,
        Poznań, 2003
    \bibitem{gausse-smth}
        Richard P. Brent,
        \textit{Multiple-precision zero-finding methods and the complexity of elementary function evaluation},
        2010,
        \url{https://arxiv.org/pdf/1004.3412.pdf}.
    \bibitem{gausse2}
        Hans Herneave,
        \textit{Gauss-Legendre Algorithm},
        accessed 20.11.2022,
        \url{https://cage.ugent.be/~hvernaev/Gauss-L.html}
    \bibitem{bog}
        D. Kincaid, W. Cheney,
        \textit{Analiza numeryczna},
        Wydawnictwo Techniczno-Naukowe,
        Warszawa, 2006
    \bibitem{ramanujan}
        Srinivasa Ramanujan,
        \textit{Collected Papers of Srinivasa Ramanujan}, 
        Cambridge University Press,
        1st edition,
        2015
\end{thebibliography}

\end{document}