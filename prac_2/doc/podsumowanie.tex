\section{Podsumowanie}

W porównaniu ~Tabeli~\ref{gauss:error} oraz ~Tabeli~\ref{house:error} widzimy niewielką różnicę na zysk algorytmu wykorzystującego transformację Householdera. Miał on ogółem większą precyzję niż algorytm wykorzystujący metodę eliminacji Gaussa. Różnica w precyzji między macierzami różnej wielkości była również mniejsza w rozkładzie $QR$. Sugeruje to nieco większa stabilność numeryczną metody Householdera.

Złożoność obu algorytmów była taka sama - oba wykonują $O(n^3)$ operacji, jednak przy metodzie Householdera potrzebujemy większej liczby zmiennych. Z drugiej strony, metoda Householdera zwraca więcej informacji o macierzy $A$ niz metoda eliminacji Gaussa.

Jeżeli potrzebujemy bardzo wysokiej precyzji obliczeń lepszym wyborem jest metoda Householdera. Jednak jeśli liczy się prostota implementacji oraz zmniejszenie ilości używanej przez algorytm pamięci, to metoda eliminacji Gaussa jest bardzo dobrym rozwiązaniem.

\koniec