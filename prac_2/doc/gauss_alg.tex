\section{Algorytm eliminacji Gaussa}

\subsection{Wyniki}

\begin{figure}[!h]\centering
\begin{tabularx}{100mm}{| >{\hsize=.20\hsize}X | >{\hsize=.3\hsize}X | >{\hsize=.3\hsize}X | >{\hsize=.25\hsize}X |}
    \hline

    \raggedleft Precyzja & $20\times20$ & $100\times100$ & $400\times400$\\

    \hline

    \raggedleft68 & 41.78 & 35.1 & 31.15\\

    \hline

    \raggedleft419 & 286.01 & 280.42 & 276.44\\

    \hline

    \raggedleft2005 & 1474.83 & 1471.28 & 1466.6\\
    \hline

\end{tabularx}
\renewcommand{\figurename}{Tabelka}
\caption{Rząd wielkości błędu metody eliminacji Gaussa.}
\label{gauss:error}
\end{figure}

W ~Tabeli~\ref{gauss:error} umieszczony został rząd wielkości normy $\|x'-x\|$. Bez względu na precyzję obliczeń większe macierze dawały nieco mniej precyzyjne wyniki. Co ciekawe, różnica ta, w okolicach $4$, ulegała jedynie niezauważalnym zmianom między różnymi precyzjami obliczeń.

\subsection{Złożoność obliczeniowa}

W zaimplementowanym przez nas algorytmie przy każdym kroku dokonujemy jednego dzielenia skalaru oraz jednego mnożenia przez skalar i odejmowania wektora o długości $n$, gdzie $n$ to szerokość naszej macierzy. Czyli za każdym razem wykonujemy $a=1+2n$ operacji. Co więcej, w algorytmie zewnętrzna pętla wykonuje się $(n-1)$ razy, natomiast wewnętrzna - $n+1-(i+1)$ razy, gdzie $i$ to jest aktualna pozycja zewnętrznej pętli. Czyli wykonujemy
$$\sum\limits_{i=1}^{n-1}(n-i)a={a(n-2)^2\over 2}={(1+2n)(n-2)^2\over 2}=O(n^3)$$
operacji w trakcie całego algorytmu.