\section{Wstęp}

Fajne podpierdalanko:

\href{https://people.inf.ethz.ch/gander/papers/qrneu.pdf}{algorytmy}

\href{https://inst.eecs.berkeley.edu/~ee127/sp21/livebook/l_lineqs_solving.html}{jak użyć do rozwiązywania równań}

\subsection{Metodologia}

W poniższej pracy zostaną porównane dwa sposoby doprowadzania macierzy kwadratowej $A$ do postaci górnotrójkątnej: metoda eliminacji Gaussa oraz rozkład $QR$ z transformacją Householdera. Oba te algorytmy zostaną wykorzystane do rozwiązywania układu równań 
$$Ax=b$$
dla odwracalnej macierzy $A$ oraz dowolnego wektora $b$. Błąd dla każdej z metod będzie obliczany jako
$$e=-\log(\|Ax-b\|).$$
Dla metody Householdera, która produkuje nam macierz ortogonalną $Q$ oraz górnotrójkątną $R$, porównane również zostaną macierze
$$A-QR$$
$$Q^TA-R$$
$$Q^TQ-I$$
poprzez wyznaczenie normy wektora $Bx$, gdzie $B$ to jedna z powyższych macierzy, z $x$ jest wektorem składającym się wyłącznie z $1$. Macierze będące wynikami powyższych przekształceń powinny być macierzami zerowymi, co wynika z przekształcenia $A=QR$ oraz ortogonalności macierzy $Q$ (wtedy $Q^{-1}=Q^T$).

Dodatkowo, sprawdzona zostanie złożoność obliczeniowa każdego z algorytmów.