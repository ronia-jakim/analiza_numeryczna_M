\section{Wstęp}

Od szkoły podstawowej przez gimnazjum (lub szkołę podstawową 2.0), liceum, przez studia i okazjonalnie po studiach wymagane od uczestników systemu szkolnictwa w Polsce wymagana jest umiejętność rozwiązywania równań. Zaczynamy od prostych, przyjemnych problemów - znajdź $x$. Można go wskazać paluszkiem, a można znaleźć jego wartość. Każdy jest w stanie to zrobić. Potem do $x$ dochodzi $y$ i powstaje chłopiec lub też układ równań. Znalezienie obu jest to zadanie trudniejsze, a czasem wręcz niemożliwe. Potem idziemy na studia i spotykamy $x_k$ i to jest moment w którym z grzecznego, ułożonego chłopca układ równań staje się hydrą o $n$ głowach, która kąsa i gryzie studenta który bez pomocy maszyny nie umie nawet pomnożyć $7\cdot13$. W takich momentach potrzebna jest interwencja boska lub maszynowa. Chciałoby się zakrzyknąć "Google, solve this set of equations for me!". Ale Google nie słucha lub nie umie. Wolfram milczy wymownie w kącie, oceniając wybory życiowe studenta. Wtedy właśnie student musi wziąć życie w swoje ręce, siąść przed komputerem, otworzyć nano i zacząć tworzyć kod, który rozwiąże tę $n$-głową hydrę i pokaże, że $x$ jest zerem.

W poniższej pracy zbierzemy nasze studenckie siły i postaramy się pokonać hydrę, trafiając jak najbliżej każdej z jej $n$ głów.

\subsection{Metodologia}

W poniższej pracy zostaną porównane dwa sposoby doprowadzania macierzy kwadratowej $A$ do postaci górnotrójkątnej: metoda eliminacji Gaussa oraz rozkład $QR$ z transformacją Householdera. Oba te algorytmy zostaną wykorzystane do rozwiązywania układu równań 
$$Ax=b$$
dla odwracalnej macierzy $A$ oraz dowolnego wektora $x$.

W ramach testowania praktycznego zastosowania naszych algorytmów, macierz $A$ będzie losowo wygenerowaną macierzą $A\in GL_n(\R)$ dla $n\in\{20,100,400\}$. Dalej losowany będzie wektor $x$ i na tej podstawie wyliczymy wektor $b$. Stosowane przez nas algorytmy dostaną jedynie macierz $A$ oraz wektor $b$, a zwrócą wektor $x'$ mający być rozwiązaniem jak wyżej. Co więcej, dla algorytmu rozkładu $QR$ jesteśmy w stanie sprawdzić poprawność obliczonej formy macierzy $A$.Błąd dla każdej z metod będzie obliczany jako
$$e=-\log(\|x'-x\|).$$

Dla metody Householdera, która produkuje macierz ortogonalną $Q$ oraz górnotrójkątną $R$, takie, że $A=QR$, porównane również zostaną macierze
$$A-QR$$
$$Q^TA-R$$
$$Q^TQ-I$$
poprzez wyznaczenie normy macierzy $B$, gdzie $B$ to jedna z powyższych macierzy. Macierze będące wynikami powyższych przekształceń powinny być macierzami zerowymi, co wynika z przekształcenia $A=QR$ oraz ortogonalności macierzy $Q$ (wtedy $Q^{-1}=Q^T$). Do obliczania normy macierzy $B$ wykorzystana będzie biblioteczna funkcja \verb+norm(A, p)+ mająca zwracać $p$-normę macierzy $A$. Ponieważ jesteśmy w przestrzeni Euklidesowej, interesować nas będzie $p=2$. 

Wszystkie obliczenia zostaną wykonane na \verb+BigFloatach+ w trzech precyzjach: $68$, $419$ oraz $2005$.